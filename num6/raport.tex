\documentclass[11pt]{extarticle}
\usepackage[a4paper, margin=1in]{geometry}
\usepackage{multicol}
\usepackage{float}
\usepackage{amsmath}
\usepackage{amsfonts}
\usepackage{lipsum}
\usepackage{mathtools}
\usepackage{cuted}
\usepackage{pgf}
\usepackage{subfigure}
\usepackage[T1]{fontenc}
\usepackage[polish]{babel}
\usepackage[utf8]{inputenc}
\author{Grzegorz Janysek}
\title{Raport - Zadanie numeryczne 5}

\begin{document}
	\maketitle

	\section{Wstęp teoretyczny}

	\pagebreak
	\section{Wyniki}
	\begin{figure}[H]
		\begin{center}
			\input{f1_u.pgf}
		\end{center}
		\caption{Interpolacja wielomianowa funkcji \( y(x)=\frac{1}{1+25x^2} \) dla jednorodnej dystrybucji \(n\) węzłów interpolacyjnych.}
		\label{f1u}
	\end{figure}
	\begin{figure}[H]
		\begin{center}
			\input{f2_u.pgf}
		\end{center}
		\caption{Interpolacja wielomianowa funkcji \( \widetilde{y}(x)=\frac{1}{1+x^2} \) dla jednorodnej dystrybucji \(n\) węzłów interpolacyjnych.}
		\label{f2u}
	\end{figure}
	\begin{figure}[H]
		\begin{center}
			\input{f1_c.pgf}
		\end{center}
		\caption{Interpolacja wielomianowa funkcji \( y(x)=\frac{1}{1+25x^2} \) dla kosinusoidalnej dystrybucji \(n\) węzłów interpolacyjnych.}
		\label{f1c}
	\end{figure}
	\begin{figure}[H]
		\begin{center}
			\input{f2_c.pgf}
		\end{center}
		\caption{Interpolacja wielomianowa funkcji \( \widetilde{y}(x)=\frac{1}{1+x^2} \) dla kosinusoidalnej dystrybucji \(n\) węzłów interpolacyjnych.}
		\label{f2c}
	\end{figure}
	\pagebreak

	Na rys. od 1 do 4 przedstawione są wyniki interpolacji funkcji \(y\) i \(\widetilde{y}\)
	\begin{align}
		y(x) = \frac{1}{1+25x^2} \\
		\widetilde{y}(x) = \frac{1}{1+x^2}
	\end{align}
	w zależności od doboru węzłów interpolacji \(x_i\), tj. ich ilości \(n\) i rodzaju dystrybucji.
	\begin{align}
		x_i &= \frac{2i}{n+1}-1 &&i=(0, 1, \dotsm, n)
		\qquad \text{dystrybucja jednoroda} \\
		x_i &= cos\left(\frac{2i+1}{2n+2}\pi\right) &&i=(0, 1, \dotsm, n) 
		\qquad \text{dystrybucja kosinusoidalna}
	\end{align}

	Porównując rys. \ref{f1u} i \ref{f2u} można zauważyć że obecność oscylacji Rungego jest silnie zależna nie tylko od ilości węzłów interpolacji ale również od interpolowanej funkcji.
	W przypadku \(y\) i dystrybucji jednorodnej (rys. \ref{f1u}) oscylacje zwiększają się wraz z \(n\), oraz są widoczne dla każdego zbadanego \(n\).
	Algorytm interpolacji sprawdził znacząco lepiej dla \(\widetilde{y}\), funkcja jest odwzorowana dokładniej już dla małego \(n\), a oscylacje Rungego nie są zauważalne.

	Zauważalną poprawę może dać odpowiedni dobór węzłów interpolacji. Analizując rys. \ref{f1c} i \ref{f2c} widać że zagęszczenie dystrybucji węzłów przy końcach przedziału interpolacji znacząco zmniejszyło oscylacje dla funkcji \(y\), oraz nie pogorszyło interpolacji \(\widetilde{y}\).

	\section{Podsumowanie}
	
\end{document}
