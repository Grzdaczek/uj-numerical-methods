\documentclass[11pt]{extarticle}
\usepackage[a4paper, margin=1in]{geometry}
\usepackage{multicol}
\usepackage{float}
\usepackage{amsmath}
\usepackage{amsfonts}
\usepackage{lipsum}
\usepackage{mathtools}
\usepackage{cuted}
\usepackage{pgf}
\usepackage{subfigure}
\usepackage[T1]{fontenc}
\usepackage[polish]{babel}
\usepackage[utf8]{inputenc}
\author{Grzegorz Janysek}
\title{Raport - Zadanie numeryczne 6}

\begin{document}
	\maketitle

	\section{Wstęp teoretyczny}

	\subsection{}
	Celem interpolacji jest znalezienie funkcji \(f\) przechodzącej przez stabelaryzowane wartości funkcji \(g\) nazywane węzłami.
	Istnieje nieskończenie wiele funkcji, takich należą nie nich węzły.
	Dobra metoda interpolacji minimalizuje błąd \(E(x)=|g(x)-f(x)|\).
	Znajduje funkcje \(f\) wiernie odwzorowującą \(g\), oraz mało kosztowną do obliczenia w danym punkcie.
	W ogólnym przypadku funkcja \(g\) nie jest znana.

	\subsection{}
	Jednym z rodzajów interpolacji jest interpolacja wielomianowa. Dla \(n\) węzłów:
	\begin{align}
		(x_0; y_0), (x_1; y_1), (x_2; y_2) \dotsm (x_n; y_n), (x_{n}; y_{n}) \quad : \quad g(x_i) = y_i
	\end{align}
	polega ona na zbudowaniu wielomianu \(f(x)\), co najwyżej \(n-1\) stopnia takiego, że \(f(x_i) = g(x_i)\)
	\begin{align}
		f(x) = a_n x^n + a_{n-1} x^{n+1} + \dotsm + a_2 x^2 + a_1 x + a_0
	\end{align}
	z wykorzystaniem wzoru Lagrange'a (\refeq{lag}).
	\begin{align}
		f(x) = \sum_{j=1}^n y_j \phi_j(x)
		\label{lag}
	\end{align}
	\begin{align}
		\phi_j(x) = \prod_{i \neq j} \frac{x-x_i}{x_j-x_i} = \frac
			{(x-x_1) \dotsm (x-x_{j-1})(x-x_{j+1}) \dotsm (x-x_n)}
			{(x_j-x_1) \dotsm (x_j-x_{j-1})(x_j-x_{j+1}) \dotsm (x_j-x_n)}
	\end{align}

	Zaletą interpolacji wielomianowej jest liniowy (w stosunku do stopnia wielomianu) koszt obliczenia wartości wielomianu, oraz możliwość analitycznego wyznaczenia jego pochodnych oraz całek.

	Intuicyjnym jest, że większa ilość węzłów interpolacji dokładniej określa wyjściową funkcję \(g\), a tym samym pozwala na jej interpolację z mniejszym błędem.
	Dlatego jednym ze sposobów polepszenia dokładności interpolacji mogło by być zagęszczenie siatki węzłów, tj. wyznaczenie dodatkowych, pośrednich wartości \(g\).
	W przypadku interpolacji wielomianowej takie działanie od pewnego momentu może mieć efekt odwrotny od oczekiwanego i skutkować powstaniem oscylacji na krańcach przedziału interpolacji, nazywanymi oscylacjami Rungego.

	\subsection{}
	Zadanie polega na znalezieniu wielomianów interpolacyjnych funkcji (5) na przedziale \(x\in[-1, 1]\) dla jednorodnej oraz kosinusoidalnej dystrybucji węzłów interpolacyjnych.
	\begin{align}
		y(x) = \frac{1}{1+25x^2} \qquad \widehat{y}(x) = \frac{1}{1+x^2}
	\end{align}

	\clearpage
	\section{Wyniki}
	\begin{figure}[H]
		\begin{center}
			\input{f1_u.pgf}
		\end{center}
		\caption{Interpolacja wielomianowa funkcji \( y(x)=\frac{1}{1+25x^2} \) dla jednorodnej dystrybucji \(n\) węzłów interpolacyjnych.}
		\label{f1u}
	\end{figure}
	\begin{figure}[H]
		\begin{center}
			\input{f2_u.pgf}
		\end{center}
		\caption{Interpolacja wielomianowa funkcji \( \widetilde{y}(x)=\frac{1}{1+x^2} \) dla jednorodnej dystrybucji \(n\) węzłów interpolacyjnych.}
		\label{f2u}
	\end{figure}
	\begin{figure}[H]
		\begin{center}
			\input{f1_c.pgf}
		\end{center}
		\caption{Interpolacja wielomianowa funkcji \( y(x)=\frac{1}{1+25x^2} \) dla kosinusoidalnej dystrybucji \(n\) węzłów interpolacyjnych.}
		\label{f1c}
	\end{figure}
	\begin{figure}[H]
		\begin{center}
			\input{f2_c.pgf}
		\end{center}
		\caption{Interpolacja wielomianowa funkcji \( \widetilde{y}(x)=\frac{1}{1+x^2} \) dla kosinusoidalnej dystrybucji \(n\) węzłów interpolacyjnych.}
		\label{f2c}
	\end{figure}
	\clearpage

	Na rys. od 1 do 4 przedstawione są wyniki interpolacji funkcji \(y\) i \(\widetilde{y}\) w zależności od doboru węzłów interpolacji, tj. ich ilości \(n\) i rodzaju dystrybucji.
	\begin{align}
		\text{dystrybucja jednorodna} \qquad
		x_i &= \frac{2i}{n+1}-1 &&i=(0, 1, \dotsm, n) \\
		\text{dystrybucja kosinusoidalna} \qquad
		x_i &= cos\left(\frac{2i+1}{2n+2}\pi\right) &&i=(0, 1, \dotsm, n)
	\end{align}

	Porównując rys. \ref{f1u} i \ref{f2u} można zauważyć, że obecność oscylacji Rungego jest silnie zależna nie tylko od ilości węzłów interpolacji ale również od interpolowanej funkcji.
	W przypadku \(y\) i dystrybucji jednorodnej (rys. \ref{f1u}) oscylacje zwiększają się wraz z \(n\), oraz są widoczne dla każdego zbadanego \(n\) i już dla \(n=17\) osiągają amplitudę przewyższającą co wartości amplitudę funkcji \(y\).
	Algorytm interpolacji sprawdził znacząco lepiej dla \(\widetilde{y}\), funkcja jest odwzorowana dokładniej już dla małego \(n\), a oscylacje Rungego nie są dostrzegalne.

	Zauważalną poprawę może dać odpowiedni dobór węzłów interpolacji. Analizując rys. \ref{f1c} i \ref{f2c} widać, że zagęszczenie dystrybucji węzłów przy końcach przedziału interpolacji znacząco zmniejszyło oscylacje dla funkcji \(y\), oraz nie pogorszyło interpolacji \(\widetilde{y}\).

	\section{Podsumowanie}
	Skuteczność interpolacji wielomianowej jest silnie zależna od interpolowanej funkcji.
	W przypadku wystąpienia oscylacji Rungego zagęszczenie węzłów w krańcach przedziału może zmniejszyć ich amplitudę zmniejszając tym samym błąd interpolacji.
	Metoda jest dobra do interpolacji funkcji, o których wiemy lub podejrzewamy, że są o charakterze wielomianowym.

\end{document}
