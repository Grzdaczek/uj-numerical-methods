\documentclass[11pt]{extarticle}
\usepackage[a4paper, margin=1in]{geometry}
\usepackage{multicol}
\usepackage{float}
\usepackage{amsmath}
\usepackage{lipsum}
\usepackage{mathtools}
\usepackage{cuted}
\usepackage[T1]{fontenc}
\usepackage[polish]{babel}
\usepackage[utf8]{inputenc}
\author{Grzegorz Janysek}
\title{Raport - Zadanie numeryczne 4}

\begin{document}
	\maketitle

	\section{Wstęp teoretyczny}
	
	\subsection{}
	Rozwiązując problem \( y=A^{-1}b\), w celu przyspieszenia obliczeń i zmniejszenia zużyciu zasobów, chcemy wykożystać strukturę \( A \).
	Jeżeli jesteśmy w stanie przedstawić problem jako:
	\begin{align}
		\label{y2}
		y &= (B + uv^T)^{-1}b \qquad \text{gdzie} \qquad A = B + uv^T
	\end{align}
	w taki sposób że złożność faktoryzacji \(B\) jest mniejsza od złożności faktoryzacji \(A\),
	możemy wykożystać wzór Shermana-Morrisona:
	\begin{align}
		(B + uv^T)^{-1} = B^{-1} - \frac{ B^{-1}uv^TB^{-1} }{ 1 + v^TB^{-1}u }
	\end{align}
	Podstawiając do (\ref{y2}) otrzymujemy
	\begin{align}
		y &= \left( B^{-1} - \frac{ B^{-1}uv^TB^{-1} }{ 1 + v^TB^{-1}u } \right)b \\[12pt]
		y &= B^{-1}b - \frac{ B^{-1}uv^TB^{-1}b }{ 1 + v^TB^{-1}u }
	\end{align}
	Zauważmy teraz że \(B\) występuje tylko w postaci \(B^{-1}\).
	Nie chcemy explicite obliczać odwrotności \(B\).
	To czego potrzebujemy to \(B^{-1}b\) oraz \(B^{-1}u\).
	\begin{align}
		p &= B^{-1}b \qquad q = B^{-1}u \\[12pt]
		y &= p - \frac{ qv^Tp }{ 1 + v^Tq }
	\end{align}
	Problem sprowadza się więc do znalezienia faktoryzacji \(B\) w celu obliczenia \(p\) i \(q\).
	

	\section{Wyniki}

	\section{Podsumowanie}

\end{document}
