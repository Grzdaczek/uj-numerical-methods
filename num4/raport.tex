\documentclass[11pt]{extarticle}
\usepackage[a4paper, margin=1in]{geometry}
\usepackage{multicol}
\usepackage{float}
\usepackage{amsmath}
\usepackage{amsfonts}
\usepackage{lipsum}
\usepackage{mathtools}
\usepackage{cuted}
\usepackage{pgf}
\usepackage{subfigure}
\usepackage[T1]{fontenc}
\usepackage[polish]{babel}
\usepackage[utf8]{inputenc}
\author{Grzegorz Janysek}
\title{Raport - Zadanie numeryczne 4}

\begin{document}
	\maketitle

	\section{Wstęp teoretyczny}

	\subsection{}
	Rozwiązując problem \( y=A^{-1}b\), w celu przyspieszenia obliczeń i zmniejszenia zużycia zasobów, chcemy wykorzystać strukturę \( A \).
	Jeżeli jesteśmy w stanie przedstawić problem jako:
	\begin{align}
		\label{y2}
		y &= (B + uv^T)^{-1}b \qquad \text{gdzie} \qquad A = B + uv^T
	\end{align}
	w taki sposób, że złożność faktoryzacji \(B\) jest mniejsza od złożności faktoryzacji \(A\),
	możemy wykorzystać wzór Shermana-Morrisona:
	\begin{align}
		(B + uv^T)^{-1} = B^{-1} - \frac{ B^{-1}uv^TB^{-1} }{ 1 + v^TB^{-1}u }
	\end{align}
	Podstawiając do (\ref{y2}) otrzymujemy:
	\begin{align}
		y &= \left( B^{-1} - \frac{ B^{-1}uv^TB^{-1} }{ 1 + v^TB^{-1}u } \right)b \\[12pt]
		y &= B^{-1}b - \frac{ B^{-1}uv^TB^{-1}b }{ 1 + v^TB^{-1}u }
	\end{align}
	Zauważmy teraz, że \(B\) występuje tylko w postaci \(B^{-1}\).
	Nie chcemy explicite obliczać odwrotności \(B\).
	To czego potrzebujemy to \(B^{-1}b\) oraz \(B^{-1}u\).
	\begin{align}
		p &= B^{-1}b \qquad q = B^{-1}u \\[12pt]
		y &= p - \frac{ qv^Tp }{ 1 + v^Tq }
	\end{align}
	Problem sprowadza się więc do znalezienia faktoryzacji \(B\) w celu obliczenia \(p\) i \(q\).

	\pagebreak
	\subsection{}
	W ćwiczeniu należy wykorzystać srukturę zadanej macierzy \(A\), i rozwiązać \(y = A^{-1}b\).
	\begin{align}
		A = \begin{bmatrix}
			10		& 8		& 1		& \dotsm	& 1		& 1		& 1 	\\
			1		& 10	& 8		& \dotsm	& 1		& 1		& 1 	\\
			1		& 1		& 10	& \dotsm	& 1		& 1		& 1 	\\
			\vdots	&\vdots	&\vdots	& \ddots	&\vdots	&\vdots	&\vdots	\\
			1		& 1		& 1		& \dotsm	& 10	& 8		& 1 	\\
			1		& 1		& 1		& \dotsm	& 1		& 10	& 8 	\\
			1		& 1		& 1		& \dotsm	& 1		& 1		& 10
		\end{bmatrix} \qquad
		b = \begin{bmatrix}
			5 \\
			5 \\
			5 \\
			5 \\
			5 \\
			\vdots \\
			5 \\
		\end{bmatrix} \\[12pt]
		A \in \mathbb{R}^{N \times N}, \quad
		b \in \mathbb{R}^{N}, \quad
		N = 50
	\end{align}
	Łatwo można zauważyć, że poza wstęgą macierz \(A\) zawiera jedynie elementy o wartośći \(1\).
	Pozwala to na zapisanie \(A\) jako sumy macierzy wstęgowej i macierzy jedynek: \(A = B + uv^T\)
	\begin{align}
		B = \begin{bmatrix}
					9		& 7	\\
			& 		9		& 7 \\
			&& 		9		& 7 \\
			&&&		\ddots	& \ddots \\
			&&&&	9		& 7 \\
			&&&&&	9		& 7 \\
			&&&&&&	9 \\
		\end{bmatrix} \qquad
		u = v = \begin{bmatrix}
			1 \\
			1 \\
			1 \\
			1 \\
			1 \\
			\vdots \\
			1 \\
		\end{bmatrix}
	\end{align}
	Z zależności (5, 6) powstają dwa układy równań z \(B\), dlatego optymalnym jest znalezienie rozkładu \(B\) i wykorzystanie go do obliczenia obu rozwiązań.
	Złożoność faktoryzacji \(LU\) macierzy wstęgowej, \textit{backward-substitution} i \textit{forward-substitution} to \(O(n)\).
	Pozostałe operacje konieczne do obliczenia \(y\) to iloczyny skalarne wektorów o złożności \(O(n)\)
	oraz mnożenie wektora przez skalar o złożności \(O(n)\).
	Z powyższego wynika, że całkowita złożność czasowa obliczenia rozwiązania wynosi \(O(n)\) w porówaniu do \(O(n^3)\) w ogólnym przypadku.

	
	\section{Wyniki}

	Wyniki obliczeń z użyciem wzoru Shermana-Morrisona zgadzają się z wynikami obliczonymi z użyciem metody biblioteki algebry komputerowej nie wykorzystującej struktury macierzy \(A\) (7).
	\begin{align}
		y = \begin{bmatrix}
			0.07525844 \\
			0.07525904 \\
			0.07525827 \\
			0.07525926 \\
			0.07525799 \\
			0.07525963 \\
			0.07525752 \\
			0.07526122 \\
			0.07525546 \\
			0.07526287 \\
			0.07525334 \\
			\vdots \\
			% 0.07526559 \\
			% 0.07524985 \\
			% 0.07527009 \\
			% 0.07524406 \\
			% 0.07527753 \\
			% 0.0752345  \\
			% 0.07528983 \\
			% 0.07521869 \\
			% 0.07531015 \\
			% 0.07519256 \\
			% 0.07534375 \\
			% 0.07514936 \\
			% 0.07539929 \\
			% 0.07507795 \\
			% 0.0754911  \\
			% 0.07495991 \\
			% 0.07564287 \\
			% 0.07476477 \\
			% 0.07589376 \\
			% 0.0744422  \\
			% 0.07630849 \\
			% 0.07390898 \\
			% 0.07699406 \\
			% 0.07302753 \\
			% 0.07812736 \\
			% 0.07157043 \\
			% 0.08000077 \\
			% 0.06916176 \\
			% 0.08309763 \\
			% 0.06518009 \\
			% 0.08821693 \\
			% 0.05859813 \\
			% 0.09667944 \\
			% 0.04771776 \\
			% 0.11066849 \\
			% 0.02973183 \\
			0.13379325 
		\end{bmatrix}
	\end{align}
	
	\begin{figure}
		\subfigure[Wykres czasu procesora]{\input{fast.pgf}}
		\subfigure[Wykres czasu procesora]{\input{slow.pgf}}
		\caption{Czas CPU w sekundach z użyciem wyżej omawianej metody (a) oraz bez jej użycia (b) w zależności od wielkości N}
	\end{figure}

	\section{Podsumowanie}
	Wzór Shermana-Morrisona jest dobrym narzędziem znajdywania odwrotności macierzy,
	pod warunkiem, że znamy odwrotność jej zaburzenia i
	jesteśmy w stanie to zaburzenie wyrazić jako produkt zewnętrzny dwóch wektorów.
	Można zyskać na złożności nawet w przypadku, gdy nie znamy odwrotności zaburzenia,
	ale jest ono łatwiejsze do obliczenia niż odwrotność samej macierzy.
	Po przekształceniu wzór może służyć jako narzędzie przyspieszające rozwiązywanie układów równań liniowych,
	zamieniając problem faktoryzacji macierzy na problem znajdywania faktoryzacji jej zaburzenia.

\end{document}
