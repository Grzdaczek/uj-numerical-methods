\documentclass[11pt]{extarticle}
\usepackage[a4paper, margin=1in]{geometry}
\usepackage{multicol}
\usepackage{float}
\usepackage{amsmath}
\usepackage{amsfonts}
\usepackage{lipsum}
\usepackage{mathtools}
\usepackage{cuted}
\usepackage{pgf}
\usepackage{subfigure}
\usepackage[T1]{fontenc}
\usepackage[polish]{babel}
\usepackage[utf8]{inputenc}
\author{Grzegorz Janysek}
\title{Raport - Zadanie numeryczne 8}

\begin{document}
	\maketitle

	\section{Wstęp teoretyczny}
	Celem całkowania numerycznego jest obliczenie zadanej istniejącej całki w efektywny sposób i ze znanym określonym błędem.
	Aby to osiągnąć można przyjąć strategie przybliżania fragmentów funkcji całkowanej \(f\) za pomocą funkcji \(g\), tj. interpolować \(f\) za pomocą \(g\).
	Funkcje interpolującą \(g\) dobiera się tak aby jej całkę dało się łatwo wyznaczyć analitycznie.

	\subsection{}
	Całkę na każdym przedział całkowania \([x_0; x_1], [x_1; x_2] \dotsm [x_{n-1}; x_n]\) można przybliżyć całką z wielomianu interpolacyjnego Lagrange'a na tym przedziale.
	Sumą ich wartości będzie całkowita wartość przybliżenia numerycznego całki \( \int_{x_0}^{x_n} f(x) dx\).
	Uzyskujemy w ten sposób metodę kwadratur Newtona-Cotesa.
	W zależności od doboru stopnia wielomianów interpolacyjnych otrzymujemy następujące kwadratury, odpowiadające kolejno stopniom wielomianów od 1 do 4:
	\begin{align}
		\text{Metoda trapezów:} \qquad &
		\int_a^b f(x)\,dx \approx \frac{b-a}{2}(f_0 + f_1) \\
		\text{Metoda Simpsona:} \qquad &
		\int_a^b f(x)\,dx \approx \frac{b-a}{6}(f_0 + 4f_1 + f_2) \\
		\text{Metoda 3/8:} \qquad &
		\int_a^b f(x)\,dx \approx \frac{b-a}{8}(f_0 + 3f_1 + 3f_2 + f_3) \\
		\text{Metoda Milne'a} \qquad &
		\int_a^b f(x)\,dx \approx \frac{b-a}{90}(7f_0 + 32f_1 + 12f_2 + 32f_3 + 7f_4)
	\end{align}

	Proces zagęszczania przedziałów całkowania powtarza się iteracyjnie do momentu, gdy kolejne wyniki są takie same jak poprzednie z dokładnością do ustalonego małego \(\epsilon\).

	\subsection{}
	Kwadratury Newtona-Cotesa nie używają wcześniej obliczonych przybliżeń w kolejnych krokach.
	Metoda Romberga łączy metodę trapezów i ekstrapolacje Richardsona.
	Na podstawie ciągu poprzednich przybliżeń całki, ekstrapoluje jej wartość.
	Może być zdefiniowana jako:
	\begin{align}
		h_n &= \frac{b-a}{2^n}, \quad 1 \leq m \leq n
	\end{align}
	\begin{align}
		R_{1,1} &= h_1 \left[ f(a) + f(b) \right] \\[0.3cm]
		R_{n,1} &= h_n \left[ f(a) + \sum_{i=1}^{2^n-1}f(a + ih_n) + f(b) \right] \\[0.3cm]
		R_{n,m} &= R_{n,m-1} + \frac{R_{n,m-1} - R_{n-1,m-1}}{4^{m-1} - 1}
	\end{align}

	Tak jak w przypadku kwadratur Newtona-Cotesa istotne jest aby nie obliczać wielokrotnie wartości \(f\) dla tego samego argumentu.
	Można zauważyć, że poprzedni krok iteracji zawiera połowę wartości \(f\) koniecznych do wyznaczenia kolejnego kroku.
	Warunkiem zakończenia iteracji jest taka sama wartość \(R_{n,n}\) i \(R_{n-1, n-1}\) z dokładnością do ustalonego małego \(\epsilon\).

	\subsection{}
	W zadaniu należy zaimplementować dwie metody całkowania.
	Wykorzystującą złożoną kwadraturę Newtona-Cotesa z metodą Simpsona oraz Metodę Romberga.
	Należy następnie obliczyć z błędem \(\epsilon = 10^{-10}\) całki:
	\begin{align}
		F_1 &\equiv \int_0^1 f_1(x)\,dx
		= \int_0^1 sin(x)\,dx \\
		F_2 &\equiv \int_0^1 \int_0^1 f_2(x, y)\,dy\,dx
		= \int_0^1 \int_0^1 ln(x^2 + y^2 + 1)\,dx\,dy
	\end{align}

	\section{Wyniki}
	\begin{table}[H]
		\centering
		\renewcommand{\arraystretch}{1.5}
		\begin{tabular}{r|c||c|c|c}
			& całka & metoda Simpsona & metoda Romberga & funkcja biblioteczna \\
			\hline
			wartość & \(F_1\) &
				\textbf{0.4596976941}413.. &
				\textbf{0.4596976941}318.. &
				\textbf{0.4596976941}318.. \\
			ilość iteracji			& \(F_1\) & 7 & 5 & - \\
			ilość wyliczeń \(f_1\)	& \(F_1\) & 129 & 33 & - \\
			\hline
			wartość & \(F_2\) &
				\textbf{0.4266771075}053.. &
				\textbf{0.4266771075}061.. &
				\textbf{0.4266771075}258.. \\
			ilość iteracji			& \(F_2\) & 1164 & 772 & - \\
			ilość wyliczeń \(f_2\)	& \(F_2\) & 65153 & 8097 & - \\
		\end{tabular}
	\end{table}

	Szacowany błąd obliczeń przez metodę biblioteczną w obu przypadkach nie przekracza \(10^{-14}\).
	Jako że całka \(F_2\) jest podwójna, dla obu badanych metod, została ona obliczona iteracyjnie. Każde obliczenie funkcji pod zewnętrzną całką wiązało się z numerycznym wyznaczeniem wartości całki wewnętrznej.
	\begin{align}
		\label{f2_lambda}
		F_2 &\equiv \int_0^1 \int_0^1 f_2(x, y)\,dy\,dx
		= \int_0^1 g(x)\,dx, \quad \text{gdzie: } g(x)
		= \int_0^1 f_2(x, y)\,dy
	\end{align}
	Zrealizowane zostało to z użyciem wyrażeń lambda.

	\section{Podsumowanie}

	Metoda Romberga osiąga zadaną dokładność przy znaczenie mniejszej ilości iteracji i co za tym idzie, mniejszej ilości wyliczeń całkowanej funkcji.
	Podczas implementacji obu metod niezwykle istotne jest aby nie obliczać ponownie znanych już wartości całkowanej funkcji.

\end{document}
