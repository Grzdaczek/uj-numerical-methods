\documentclass[11pt]{extarticle}
\usepackage[a4paper, margin=1in]{geometry}
\usepackage{multicol}
\usepackage{float}
\usepackage{amsmath}
\usepackage{amsfonts}
\usepackage{lipsum}
\usepackage{mathtools}
\usepackage{cuted}
\usepackage{pgf}
\usepackage{subfigure}
\usepackage[T1]{fontenc}
\usepackage[polish]{babel}
\usepackage[utf8]{inputenc}
\author{Grzegorz Janysek}
\title{Raport - Zadanie numeryczne 8}

\begin{document}
	\maketitle

	\section{Wstęp teoretyczny}
	Celem całkowania numerycznego jest obliczenie zadanej istniejącej całki w efektywny sposób i ze znanym określonym błędem.
	Aby to osiągnąć można przyjąć strategie przybliżania fragmentów funkcji całkowanej \(f\) za pomocą funkcji \(g\), tj. interpolować \(f\) za pomocą \(g\).
	Funkcje interpolującą \(g\) dobiera się tak aby jej całkę dało się łatwo wyznaczyć analitycznie.

	\subsection{}
	Całkę na każdym przedział całkowania \([a; x_1], [x_1; x_2] \dotsm [x_n; b]\) można przybliżyć całką z wielomianu interpolacyjnego Lagrange'a na tym przedziale.
	Sum ich wartości będzie całkowita wartość przybliżenia numerycznego całki \( \int_a^b f(x) dx\).
	Uzyskujemy w ten sposób metodę kwadratur Newtona-Cotesa.
	W zależności od doboru stopnia wielomianów interpolacyjnych otrzymujemy następujące kwadratury, odpowiadające kolejno stopniom wielomianów od 1 do 4:
	\begin{align}
		\text{Metoda trapezów:} \qquad &
		\int_a^b f(x)\,dx \approx \frac{b-a}{2}(f_0 + f_1) \\
		\text{Metoda Simpsona:} \qquad &
		\int_a^b f(x)\,dx \approx \frac{b-a}{6}(f_0 + 4f_1 + f_2) \\
		\text{Metoda 3/8:} \qquad &
		\int_a^b f(x)\,dx \approx \frac{b-a}{8}(f_0 + 3f_1 + 3f_2 + f_3) \\
		\text{Metoda Milne'a} \qquad &
		\int_a^b f(x)\,dx \approx \frac{b-a}{90}(7f_0 + 32f_1 + 12f_2 + 32f_3 + 7f_4)
	\end{align}
	

	\section{Wyniki}
	
	\section{Podsumowanie}

\end{document}
