\documentclass[11pt]{extarticle}
\usepackage[a4paper, margin=1in]{geometry}
% \usepackage[a4paper]{geometry}
\usepackage{multicol}
\usepackage{float}
\usepackage{amsmath}
\usepackage{lipsum}
\usepackage{mathtools}
\usepackage{cuted}
\usepackage[T1]{fontenc}
\usepackage[polish]{babel}
\usepackage[utf8]{inputenc}
\author{Grzegorz Janysek}
\title{Raport - Zadanie numeryczne 2}

\begin{document}
	\maketitle

	\section{Wstęp teoretyczny}

	Problem obliczneia \( f(x) \) za pomocą metod numerycznych,
	można przedstawić jako znalezienie \( f(\overline{x} ) \), gdzie
	\( \overline{x} = x + \epsilon\) jest reprezentacją argumentu na maszynie numerycznej.
	\( \epsilon \) jest błędem numerycznym tej reprezentacji mogącym być dobrze oszacowanym od góry.
	Chcąc oszacować błąd \( E=|f(x) - f(\overline{x})|\), należy zadać pytanie: jaki wpływ na obliczenia ma \( \epsilon \).
	A dokładniej, jak zmiana wartości \( \epsilon \) wpływa na zmianę wartości \( f(\overline{x}) \).
	
	Problem nazywamy numerycznie dobrze uwarunkowanym,
	jeśli niewielkie zmiany \( \epsilon \) nie zmianiają znacząco rozwiązania problemu.
	W sytuacji gdy układ jest podatny na zabużenia; jest źle uwarunkowany,
	wyniki obliczeń mogą być zdominowane przez błędy, w konsekwęcji będąc bezwartościowymi.
	Przedstawia to istotność miary uwarunkowania, bez której poprawne zinterpretowanie rozwiązania jest nie możliwe. 
	
	Dla układów równań liniowych taką miarą jest współczynnkik uwarunkowania macierzy, będący własnością danej macierzy.
	Ćwiczenie skupia się na interpretacji różnicy wartości rozwiązań równań dla podanych macierzy.
	Zadane są dwie symetryczne macierze \( A_1 \) i \( A_2 \), oraz wektory \( b \) i \( b' \).
	Należy znaleźć \( \Delta_i \equiv ||y_i - y_i'||_2 \) gdzie: \( A_i y_i = b\) oraz \( A_i y_i' = b' \) dla \( i \in \{1, 2\}\)
	
	\begin{equation}
		A_1 = \begin{pmatrix}
			2.40827208 & -0.36066254 & 0.80575445 & 0.46309511 & 1.20708553 \\
			-0.36066254 & 1.14839502 & 0.02576113 & 0.02672584 & -1.03949556 \\
			0.80575445 & 0.02576113 & 2.45964907 & 0.13824088 & 0.0472749 \\
			0.46309511 & 0.02672584 & 0.13824088 & 2.05614464 & -0.9434493 \\
			1.20708553 & -1.03949556 & 0.0472749 & -0.9434493 & 1.92753926 \\
		\end{pmatrix}
	\end{equation}
	\begin{equation}
		A_2 = \begin{pmatrix}
			2.61370745 & -0.6334453 & 0.76061329 & 0.24938964 & 0.82783473 \\
			-0.6334453 & 1.51060349 & 0.08570081 & 0.31048984 & -0.53591589 \\
			0.76061329 & 0.08570081 & 2.46956812 & 0.18519926 & 0.13060923 \\
			0.24938964 & 0.31048984 & 0.18519926 & 2.27845311 & -0.54893124 \\
			0.82783473 & -0.53591589 & 0.13060923 & -0.54893124 & 2.6276678 \\
		\end{pmatrix}
	\end{equation}

	\begin{multicols}{2}
		\begin{equation}
			b = \begin{pmatrix}
				5.40780228 \\
				3.67008677 \\
				3.12306266 \\
				-1.11187948 \\
				0.54437218 \\
			\end{pmatrix}
		\end{equation}

		\begin{equation}
			b' = b + \begin{pmatrix}
				10^{-5} \\
				0 \\
				0 \\
				0 \\
				0 \\
			\end{pmatrix}
		\end{equation}
	\end{multicols}

	\pagebreak

	\section{Wyniki}

	\begin{multicols}{2}
		\begin{equation}
			y_1 = \begin{pmatrix}
				3.28716602 \\
				3.8029998 \\
				0.25146854 \\
				-1.57875474 \\
				-0.50410395 \\
			\end{pmatrix}
		\end{equation}

		\begin{equation}
			y_1' = \begin{pmatrix}
				16.74173331 \\
				-14.06233582 \\
				-2.70495914 \\
				-15.57494944 \\
				-25.34234554 \\
			\end{pmatrix}
		\end{equation}
	\end{multicols}

	\begin{multicols}{2}
		\begin{equation}
			y_2 = \begin{pmatrix}
				3.18374857 \\
				3.94032033 \\
				0.27419287 \\
				-1.47117406 \\
				-0.31318674 \\
			\end{pmatrix}
		\end{equation}

		\begin{equation}
			y_2' = \begin{pmatrix}
				3.18375389 \\
				3.94032237 \\
				0.27419131 \\
				-1.47117514 \\
				-0.31318814 \\
			\end{pmatrix}
		\end{equation}
	\end{multicols}

	\begin{equation}
		\Delta_1 \equiv ||y_1 - y_1'||_2 = 36.35612430090815 \approx 3.64 * 10^{1}
	\end{equation}

	\begin{equation}
		\Delta_2 \equiv ||y_2 - y_2'||_2 = 6.16673946544916 * 10^{-6} \approx 6.17 * 10^{-6}
	\end{equation}

	Wiedząc że obie macierze są symetryczne i rzeczywiste,
	można dla nich obliczyć współczynniki uwarunkowania ze wzoru \refeq{eq.wsp_uw_kappa},
	gdzie \( \lambda_i \) oznaczają wartości własne macierzy.

	\begin{equation}
		\label{eq.wsp_uw_kappa}
		\kappa = \frac{ \max_{i} |\lambda_i| }{  \min_{i} |\lambda_i|  }
	\end{equation}

	\begin{equation}
		\kappa_1 = 39295747.864306 \approx 3.93 * 10^7
	\end{equation}
	
	\begin{equation}
		\kappa_2 = 4.000000024553183 \approx 4.00 * 10^0
	\end{equation}

	% u wrażliwości układów na zabirzenia
	% o to że jest to ważne zagadnienie i w czym ono odgrywa dużą rolę

	\section{Podsumowanie}

	Warotść \( \Delta_1 \) jest 7 rzędów wielkości większa od \( \Delta_2 \),
	pokazuje to znacznie większą czułość na błąd wyrazu wolnego równania z macierzą \( A_1 \) od równania z macierzą \( A_2 \).
	Układ równań z macierzą \( \Delta_1 \) jest stosunkowo źle uwarunkowany.
	Potwierdzają to wartości współczynnkików \( \kappa_1 \) i \( \kappa_2 \).
	Znając niepewność wyrazu wolnego współczynnkik uwarunkowania pozwala określić precyzję wyniku,
	w efekcie umożliwiając jego poprawne zinterpretowanie.

\end{document}
