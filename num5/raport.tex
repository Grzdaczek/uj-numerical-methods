\documentclass[11pt]{extarticle}
\usepackage[a4paper, margin=1in]{geometry}
\usepackage{multicol}
\usepackage{float}
\usepackage{amsmath}
\usepackage{amsfonts}
\usepackage{lipsum}
\usepackage{mathtools}
\usepackage{cuted}
\usepackage{pgf}
\usepackage{subfigure}
\usepackage[T1]{fontenc}
\usepackage[polish]{babel}
\usepackage[utf8]{inputenc}
\author{Grzegorz Janysek}
\title{Raport - Zadanie numeryczne 5}

\begin{document}
	\maketitle

	\section{Wstęp teoretyczny}

	\subsection{}
	Rozwiązywanie układów równań liniowych za pomocą metod iteracyjnych polega na znalezieniu przybliżenia dokładnego wyniku na drodze skończonej liczby iteracji poczynając od dowolnie wybranego wektora.
	Uzyskuje się to powtarzając obliczenia i znajdując z każdą kolejną iteracją lepsze przybliżenie rozwiązania równiania.
	
	Wykonanie rozkładu macierzy w arytmetyce dokładnej pozwala na obliczenie ścisłego wyniku,
	natomiast w przypadku omawianych metod iteracyjnych ścisły wynik musiał by być efektem iteracji, których ilość dąży do nieskończoności.
	W praktyce dokładność przybliżenia, ograniczoną typem danch, wybiera się dowolnie.
	Mniejszy błąd przybliżenia wyniku równiania uzyskiwany jest większąliczbą iteracji.

	Szybkością zbiegania metody iteracyjnej określa się tempo z jakim maleje błąd przybliżenia wyniku z każdą kolejną iteracją.
	Zakładając stałą złożoność iteracji, metoda która dla danego problemu zbiega się szybciej będzie lepsza.
	\subsection{}
	Porównywane dalej metody to metoda \textit{Jacobiego} i metoda \textit{Gaussa-Seidela} należą do ogólnej kategorii metod iteracyjnych:
	\begin{align}
		Mx^{(k+1)} = Nx^{(k)} + b
	\end{align}
	Gdzie indeks \(k\) oznacza numer iteracji.
	Dla równania \(Ax = b\), \(A = M - N\) jest podziałem wybranym w różny sposób w zależności od metody iteracyjnej.
	Podział dla metody \textit{Jacobiego} (\ref{splitting-1}) i \textit{Gaussa-Seidela} (\ref{splitting-2})
	\begin{align}
		\label{splitting-1}
		&A = D + (L + U)	&&M=D		&&N = - (L + U) \\
		\label{splitting-2}
		&A = (D + L) + U	&&M=D + L	&&N = - U
	\end{align}
	Z powyższych wzorów można wyprowadzić wyrażenia na \(i\)-ty element wektora w \(k+1\) iteracji odpowiednio dla obu metod:
	\begin{align}
		x_i^{(k+1)} &= \frac{1}{a_{ii}} \left( b_i - \sum_{j=1}^{i-1} a_{ij} x_{i}^{(k)} - \sum_{j=i+1}^{N} a_{ij} x_{i}^{(k)} \right) \\
		x_i^{(k+1)} &= \frac{1}{a_{ii}} \left( b_i - \sum_{j=1}^{i-1} a_{ij} x_{i}^{(k+1)} - \sum_{j=i+1}^{N} a_{ij} x_{i}^{(k)} \right)
	\end{align}

	\subsection{}
	W ćwiczeniu należy rozwiązać układ równań \(Ax = b\) za pomocą omawianych metod iteracyjnych i porównać ich tempo zbieżności do rozwiązania.
	\begin{align}
		B = \begin{bmatrix}
									3		& 1		& 0.2 \\
							1 		& 3		& 1		& 0.2 \\
					0.2		& 1		& 3		& 1		& 0.2 \\
			& 		\ddots	&\ddots &\ddots &\ddots &\ddots \\
			&&		0.2		& 1		& 3		& 1		& 0.2 \\
			&&&		0.2		& 1		& 3		& 1	\\
			&&&&	0.2		& 1		& 3
		\end{bmatrix} \qquad
		b = \begin{bmatrix}
			1 \\
			2 \\
			3 \\
			\vdots \\
			N-1 \\
			N \\
		\end{bmatrix}
	\end{align}
	
	\section{Wyniki}

	\begin{figure}[H]
		\begin{align}
			x = \begin{bmatrix}
				0.17126009 \\
				0.37523974 \\
				0.55489993 \\
				0.74060385 \\
				0.92602310 \\
				1.11108743 \\
				1.29629727 \\
				1.48148292 \\
				1.66666609 \\
				1.85185195 \\
				2.03703705 \\
				% 2.22222221 \\
				% 2.40740741 \\
				% 2.59259259 \\
				% 2.77777778 \\
				% 2.96296296 \\
				% 3.14814815 \\
				% 3.33333333 \\
				% 3.51851852 \\
				% 3.70370370 \\
				% 3.88888889 \\
				% 4.07407407 \\
				% 4.25925926 \\
				% 4.44444444 \\
				% 4.62962963 \\
				% 4.81481481 \\
				% 5.00000000 \\
				% 5.18518519 \\
				% 5.37037037 \\
				% 5.55555556 \\
				% 5.74074074 \\
				% 5.92592593 \\
				% 6.11111111 \\
				% 6.29629630 \\
				% 6.48148147 \\
				% 6.66666672 \\
				% 6.85185177 \\
				% 7.03703673 \\
				% 7.22222479 \\
				% 7.40739955 \\
				% 7.59259499 \\
				% 7.77787115 \\
				% 7.96249685 \\
				% 8.14907400 \\
				% 8.33522653 \\
				% 8.49740196 \\
				% 8.77672533 \\
				% 8.83013724 \\
				% 8.37619757 \\
				\vdots \\
				13.2859250 \\ 
			\end{bmatrix}
		\end{align}
		\caption{Rozwiązanie równania obliczone za pomocą metody Jacobiego}
	\end{figure}

	\begin{figure}[H]
		\begin{align}
			x = \begin{bmatrix}
				0.17126009 \\
				0.37523974 \\
				0.55489993 \\
				0.74060385 \\
				0.92602310 \\
				1.11108743 \\
				1.29629727 \\
				1.48148292 \\
				1.66666609 \\
				1.85185195 \\
				2.03703705 \\
				% 2.22222221 \\
				% 2.40740741 \\
				% 2.59259259 \\
				% 2.77777778 \\
				% 2.96296296 \\
				% 3.14814815 \\
				% 3.33333333 \\
				% 3.51851852 \\
				% 3.70370370 \\
				% 3.88888889 \\
				% 4.07407407 \\
				% 4.25925926 \\
				% 4.44444444 \\
				% 4.62962963 \\
				% 4.81481481 \\
				% 5.00000000 \\
				% 5.18518519 \\
				% 5.37037037 \\
				% 5.55555556 \\
				% 5.74074074 \\
				% 5.92592593 \\
				% 6.11111111 \\
				% 6.29629630 \\
				% 6.48148147 \\
				% 6.66666672 \\
				% 6.85185177 \\
				% 7.03703673 \\
				% 7.22222479 \\
				% 7.40739955 \\
				% 7.59259499 \\
				% 7.77787115 \\
				% 7.96249685 \\
				% 8.14907400 \\
				% 8.33522653 \\
				% 8.49740196 \\
				% 8.77672533 \\
				% 8.83013724 \\
				% 8.37619757 \\
				\vdots \\
				13.2859250 \\ 
			\end{bmatrix}
		\end{align}
		\caption{Rozwiązanie równania obliczone za pomocą metody Gaussa-Seidela}
	\end{figure}

	\begin{figure}[H]
		\begin{center}
			\input{chart.pgf}
		\end{center}
		\caption{Błąd \(E\) od ilości iteracji \(k\) dla metod \textit{Jacobiego} i \textit{Gaussa-Seidela}}
	\end{figure}

	Wykres przedstawia wartość błędu bezwzględnego \(E(k) = |x^{(k)} - x|\) dla obu metod.
	Wartość \(x\) została obliczona za pomocą biblioteki numerycznej.
	Dla zadanej macierzy \(A\) metoda \textit{Gaussa-Seidela} zbiega się szybciej, ponieważ w \(k+1\) kroku iteracji do obliczenia danej składowej wektora \(x^{(k+1)}\) oprócz \(x^{(k)}\), wykorzystuje ona obliczone już w poprzednich krokach danej iteracji składowe tego \(x^{(k+1)}\), natomiast metoda \textit{Jacobiego} używa z wartości tylko z \(x^{(k)}\).
	Warunkiem zakończenia iteracji w obu przypadkach było zrównanie się norm przybliżeń z kroku poprzedniego i następnego z dokładnością do \(\epsilon = 10^{-12}\)

	\section{Podsumowanie}
	Przewaga metod iteracyjnych jest widoczna gdy, rozwiązanie równiania za pomocą faktoryzacji macierzy staje się zbyt kosztowne.
	
	Dla gęstej macierzy \(A\) złożoność rozkładu to \(O(n^3)\).
	Złożoność pojedynczej iteracji dla takiej macierzy to \(O(n^2)\), stąd złożoność metody iteracyjnej dla \(k\) iteracji \(O(k*n^2)\).
	Jeżeli \(n\) jest duże, i z przyczyn praktycznych nie możliwa jest faktoryzacja, metody iteracyjne pozwalają na uzyskanie przybliżenia.
	Mozna je poprawić w kolejnych krokach iteracji w przypadku nie wystarczającej dokładności, przez co są bardziej plastyczne od rozkładu.

	W porównywanych metodach iteracujnych istotne jest, aby wykorzystać strukturę macierzy i nie iterować po znanych elementach zerowych, co w równaniu z zadania pozwala to na osiągnięcie liniowej złożoności pojedynczej iteracji.

\end{document}
