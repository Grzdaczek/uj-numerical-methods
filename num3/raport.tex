\documentclass[11pt]{extarticle}
\usepackage[a4paper, margin=1in]{geometry}
% \usepackage[a4paper]{geometry}
\usepackage{multicol}
\usepackage{float}
\usepackage{amsmath}
\usepackage{lipsum}
\usepackage{mathtools}
\usepackage{cuted}
\usepackage[T1]{fontenc}
\usepackage[polish]{babel}
\usepackage[utf8]{inputenc}
\author{Grzegorz Janysek}
\title{Raport - Zadanie numeryczne 3}

\begin{document}
	\maketitle

	\section{Wstęp teoretyczny}
	
	\subsection{}

	W ogólnym przypadku złożoność numerycznego rozwiązywania układów równań z liniowych to \(O(n^3)\).
	Daje się ją jednak znacząco zmniejszyć stosując algorytmy wykożystujące strukturę macierzy.
	Zadanie skupia się na metodzie dla szczególego rodzaju kwadratowej macjeży żadkiej: macieży wstęgowej (in. pasmowej). 
	Charakteryzuje się ona tym że poza główną diagonalą i wstęgą wokół niej wszystkie elementy są zerowe.
	
	\begin{align}
		\begin{bmatrix}
								\bullet	&	\circ	&	\circ	\\
					\circ	&	\bullet	&	\circ	&	\circ	\\
			&		\circ	&	\bullet	&	\circ	&	\circ	\\
			&&		\ddots	&	\ddots	&	\ddots	&	\ddots	\\
			&&&		\circ	&	\bullet	&	\circ	&	\circ	\\
			&&&&	\circ	&	\bullet	&	\circ	&	\circ	\\
			&&&&&	\circ	&	\bullet &	\circ	\\
			&&&&&&	\circ	&	\bullet	&	\\
		\end{bmatrix}
	\end{align}
	
	\subsection{}

	Przechowywanie macieży w pamięci w ogólnym przypadku zajmuje \(x*n^2\), gdzie \(x\) to rozmiar typu danych.
	Do zapisania macierzy z zadania dla typu danych \emph{float64} konieczne by było: 

	\begin{align}
		\text{dla } N=100 &:& 8\text{B}*100^2 &= 80\text{kB} \\
		\text{dla } N=1,000 &:& 8\text{B}*1,000^2 &= 8\text{MB} \\
		\text{dla } N=10,000 &:& 8\text{B}*10,000^2 &= 800\text{MB}
	\end{align}

	Dla ogólnej macierzy wstęgowej większość elementów to zera na znanych pozycjach poza wstęgą.
	Pozwala to na optymalizację polegającą na przechowywaniu jedynie wstęgi, 
	w postaci np. tablicy dwuwymiarowej \(a\times n\) której rozmiar to \(x*a*n\), gdzie \(a\) to szerokość wstęgi.

	\begin{align}
		\text{dla } N=100 &:& 8\text{B}*4*100 &= 3.2\text{kB} \\
		\text{dla } N=1,000 &:& 8\text{B}*4*1,000 &= 32\text{kB} \\
		\text{dla } N=10,000 &:& 8\text{B}*4*10,000 &= 320\text{kB}
	\end{align}

	\subsection{}

	Faktoryzacja \(LU\) zachowuje strukturę macieży pasmowej.
	Wiedząc to można pominąć obliczanie elementów macieży \(L\) oraz \(U\) poza wstęgą, ponieważ z powyższego wynika że będą one zerowe.
	Ogranicza to konieczność obliczeń tylko dla elemetów wstęgi, których ilosć jest rzędu \(n\).
	Dodatkowo ilość elementów sumy koniecznych do obliczania wartości elementu w \(L\) lub w \(U\) pozostaje nie większa niż szerokość wstęgi.
	Wykożystując to można zmniejszyć złożoność rozkładu dla macieży wstęgowej do \(O(n)\).
	W połączeniu z \emph{back-substitution} i \emph{forward-substitution} których złożoność wynosi \(O(n)\),
	otrzymujemy całkowitą złożoność rozwiązywania układów równań liniowych z macieżą wstęgową wynoszącą \(O(n)\)

	\pagebreak

	\section{Wyniki}
	
	\section{Podsumowanie}

	Podczas rozwiązywania układów równań liniowych kluczowa jest znajomość struktury macieży.
	Pozwala to na dobranie odpowiedniego sposobu przechowywania w pamięci, oraz algorytmu faktoryzacji,
	potencjalnie dramatycznie zmniejszając zużycie pamięci oraz czas procesora.

\end{document}
