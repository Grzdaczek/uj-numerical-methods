\documentclass[11pt]{extarticle}
\usepackage[a4paper, margin=1in]{geometry}
\usepackage{multicol}
\usepackage{float}
\usepackage{amsmath}
\usepackage{lipsum}
\usepackage{mathtools}
\usepackage{cuted}
\usepackage[T1]{fontenc}
\usepackage[polish]{babel}
\usepackage[utf8]{inputenc}
\author{Grzegorz Janysek}
\title{Raport - Zadanie numeryczne 3}

\begin{document}
	\maketitle

	\section{Wstęp teoretyczny}
	\subsection{}
	W ogólnym przypadku złożoność numerycznego rozwiązywania układów równań liniowych to \(O(n^3)\).
	Złożoność daje się jednak znacząco zmniejszyć stosując algorytmy wykorzystujące strukturę macierzy.
	Zadanie skupia się na metodzie dla szczególego rodzaju kwadratowej macierzy rzadkiej: macierzy wstęgowej (in. pasmowej). 
	Charakteryzuje się ona tym, że poza główną diagonalą i wstęgą wokół niej, wszystkie elementy są zerowe.
	\begin{align}
		\begin{bmatrix}
								\bullet	&	\circ	&	\circ	\\
					\circ	&	\bullet	&	\circ	&	\circ	\\
			&		\circ	&	\bullet	&	\circ	&	\circ	\\
			&&		\ddots	&	\ddots	&	\ddots	&	\ddots	\\
			&&&		\circ	&	\bullet	&	\circ	&	\circ	\\
			&&&&	\circ	&	\bullet	&	\circ	&	\circ	\\
			&&&&&	\circ	&	\bullet &	\circ				\\
			&&&&&&	\circ	&	\bullet	&						\\
		\end{bmatrix}
	\end{align}
	
	\subsection{}
	Przechowywanie macierzy w pamięci w ogólnym przypadku zajmuje \(x*n^2\), gdzie \(x\) to rozmiar typu danych.
	Do zapisania macierzy dla typu danych \emph{float64} konieczne by było: 
	\begin{align}
		\text{dla } n=100 &:& 8\text{B}*100^2 &= 80\text{kB} \\
		\text{dla } n=1,000 &:& 8\text{B}*1,000^2 &= 8\text{MB} \\
		\label{rozmiar_og_N10000}
		\text{dla } n=10,000 &:& 8\text{B}*10,000^2 &= 800\text{MB}
	\end{align}

	Dla ogólnej macierzy wstęgowej większość elementów to zera, na znanych pozycjach poza wstęgą.
	Pozwala to na optymalizację polegającą na przechowywaniu jedynie wstęgi w postaci np. tablicy dwuwymiarowej \(a\times n\),
	której rozmiar to \(x*a*n\), gdzie \(x\) to rozmiar typu danych, \(a\) to szerokość wstęgi.
	Na zapis w ten sposób potrzeba znacząco mniej pamięci. Zakładając szerokość wstęgi \(a=4\) (jak dla macierzy w zadaniu) mamy:
	\begin{align}
		\text{dla } n=100 &:& 8\text{B}*4*100 &= 3.2\text{kB} \\
		\text{dla } n=1,000 &:& 8\text{B}*4*1,000 &= 32\text{kB} \\
		\label{rozmiar_diag_N10000}
		\text{dla } n=10,000 &:& 8\text{B}*4*10,000 &= 320\text{kB}
	\end{align}

	\subsection{}
	Faktoryzacja \(LU\) zachowuje strukturę macierzy pasmowej.
	Wiedząc to można pominąć obliczanie elementów macierzy \(L\) oraz \(U\) poza wstęgą, ponieważ z powyższego wynika, że będą one zerowe.
	Ogranicza to konieczność obliczeń tylko dla elemetów wstęgi, których ilosć jest rzędu \(n\).	
	Dodatkowo ilość elementów sumy koniecznych do obliczania wartości elementu w \(L\) lub w \(U\) pozostaje nie większa niż szerokość wstęgi.
	Wykorzystując to można zmniejszyć złożoność rozkładu dla macierzy wstęgowej do \(O(n)\).
	
	Otrzymujemy przekształcenia (\ref{uij}, \ref{lij}) jawnych wzorów na wartości elementów \(L\) i \(U\),
	gdzie \(W\) to zbiór elementów należących do wstęgi
	% gdzie \(p\) odległością od górnej krawędzi wstęgi elemenu \(u_{ij}\), a \(q\) jest odległością od lewek krawędzi wstęgi elemenu \(l_{ij}\)
	\begin{align}
		\label{uij}
		u_{ij} &= a_{ij} - \sum_{k \in P} l_{ik}u_{kj}									&P = \{ k:  k<i \land l_{ik} \in W \land u_{kj} \in W \} \\
		\label{lij}
		l_{ij} &= \frac{1}{u_{jj}} \left( a_{ij} - \sum_{k \in Q} l_{ij}u_{kj} \right)  &Q = \{ k: k<j \land l_{ik} \in W \land u_{kj} \in W \} 
	\end{align}

	W połączeniu z \emph{backward-substitution} i \emph{forward-substitution} których złożoność dla macierzy wstęgowej wynosi \(O(n)\),
	otrzymujemy całkowitą złożoność rozwiązywania układów równań liniowych z macieżą wstęgową wynoszącą \(O(n)\)

	\subsection{}
	W ćwiczeniu należy znaleźć \(y = A^{-1}x\) dla \(N=100\) gdzie
	\begin{align}
		A = \begin{bmatrix}
							  1.2		& \frac{0.1}{1}		& \frac{0.4}{1^2} 		\\[6pt]
					0.2		& 1.2		& \frac{0.1}{2}		& \frac{0.4}{2^2} 		\\[6pt]
			&		0.2		& 1.2		& \frac{0.1}{3}		& \frac{0.4}{3^2} 		\\[6pt]
			&&		\ddots	& \ddots	& \ddots			& \ddots 				\\[6pt]
			&&&		0.2		& 1.2		& \frac{0.1}{N-2}	& \frac{0.4}{(N-2)^2} 	\\[6pt]
			&&&&	0.2		& 1.2		& \frac{0.1}{N-1}							\\[6pt]
			&&&&&	0.2		& 1.2
		\end{bmatrix} &&
		x = \begin{bmatrix}
			1 \\[6pt]
			2 \\[6pt]
			3 \\[6pt]
			4 \\[6pt]
			5 \\[6pt]
			\vdots \\[6pt]
			N
		\end{bmatrix}
	\end{align}

	\section{Wyniki}

	Wyniki pokrywają się z oczekiwanymi uzyskanymi za pomocą biblioteki numerycznej.
	\begin{align}
		\setlength{\jot}{10pt}
		det(A) = 78240161.00959387 \\[20pt]
		y = \begin{bmatrix}
			0.03287133486041399 \\
			1.339622798096375 \\
			2.066480295894664 \\
			2.825543605175336 \\
			3.557571715528883 \\
			4.284492868897645 \\
			5.00721018451999 \\
			\vdots \\
			% 5.727664002754518 \\
			% 6.446615582748809 \\
			% 7.164554400995276 \\
			% 7.881773878242026 \\
			% 8.598465868371878 \\
			% 9.314759799907844 \\
			% 10.030746230199036 \\
			% 10.746490321152768 \\
			% 11.46204012796359 \\
			% 12.177431844626687 \\
			% 12.892693237901542 \\
			% 13.60784595684208 \\
			% 14.322907124390252 \\
			% 15.03789045794619 \\
			% 15.752807073551208 \\
			% 16.467666073000725 \\
			% 17.182474979167374 \\
			% 17.897240063340146 \\
			% 18.611966594532937 \\
			% 19.32665903159678 \\
			% 20.041321172855753 \\
			% 20.755956273816828 \\
			% 21.47056714061568 \\
			% 22.18515620483152 \\
			% 22.899725583859315 \\
			% 23.61427712998635 \\
			% 24.328812470561147 \\
			% 25.0433330410833 \\
			% 25.757840112626393 \\
			% 26.472334814693667 \\
			% 27.186818154368854 \\
			% 27.901291032443737 \\
			% 28.615754257064278 \\
			% 29.33020855532933 \\
			% 30.04465458319117 \\
			% 30.75909293394065 \\
			% 31.473524145507586 \\
			% 32.1879487067645 \\
			% 32.902367062989086 \\
			% 33.61677962061327 \\
			% 34.33118675136514 \\
			% 35.045588795892535 \\
			% 35.75998606694211 \\
			% 36.474378852156384 \\
			% 37.18876741654113 \\
			% 37.90315200464761 \\
			% 38.61753284250725 \\
			% 39.331910139350974 \\
			% 40.04628408914067 \\
			% 40.76065487193609 \\
			% 41.47502265511775 \\
			% 42.189387594482916 \\
			% 42.90374983523002 \\
			% 43.6181095128443 \\
			% 44.33246675389621 \\
			% 45.04682167676243 \\
			% 45.76117439227791 \\
			% 46.47552500432681 \\
			% 47.18987361037867 \\
			% 47.904220301975755 \\
			% 48.618565165176626 \\
			% 49.332908280960545 \\
			% 50.047249725596565 \\
			% 50.76158957098093 \\
			% 51.47592788494589 \\
			% 52.19026473154275 \\
			% 52.904600171301595 \\
			% 53.6189342614698 \\
			% 54.33326705623164 \\
			% 55.04759860691019 \\
			% 55.761928962153874 \\
			% 56.47625816810818 \\
			% 57.19058626857465 \\
			% 57.90491330515779 \\
			% 58.61923931740096 \\
			% 59.33356434291259 \\
			% 60.04788841748285 \\
			% 60.76221157519233 \\
			% 61.47653384851288 \\
			% 62.1908552684013 \\
			% 62.9051758643867 \\
			% 63.61949566465192 \\
			% 64.33381469610926 \\
			% 65.04813298447127 \\
			% 65.76245055431694 \\
			% 66.47676742915336 \\
			% 67.19108363147355 \\
			% 67.9053991828134 \\
			% 68.61971410401004 \\
			% 69.33402833257784 \\
			% 70.0483379441879 \\
			% 70.7650588638003 \\
			71.53915685603329
		\end{bmatrix}
	\end{align}
	
	\pagebreak
	\section{Podsumowanie}

	Podczas rozwiązywania układów równań liniowych bardzo istotna jest znajomość struktury macierzy.
	Pozwala ona na dobranie odpowiedniego sposobu przechowywania macierzy w pamięci;
	dla rozpatrywanego przypadku \(n=10,000\) (\ref{rozmiar_og_N10000}, \ref{rozmiar_diag_N10000})
	optymalizacja reprezentacji pozwala na zmniejszenia zapotrzebowania na pamięć o \(96\%\).
	Oraz umożliwia wybranie algorytmu faktoryzacji o najmniejszej złożoności,
	potencjalnie zmniejszając czas procesora \(n^2\) razy.

\end{document}
