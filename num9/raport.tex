\documentclass[11pt]{extarticle}
\usepackage[a4paper, margin=1in]{geometry}
\usepackage{multicol}
\usepackage{float}
\usepackage{amsmath}
\usepackage{amsfonts}
\usepackage{lipsum}
\usepackage{mathtools}
\usepackage{cuted}
\usepackage{pgf}
\usepackage{subfigure}
\usepackage[T1]{fontenc}
\usepackage[polish]{babel}
\usepackage[utf8]{inputenc}
\author{Grzegorz Janysek}
\title{Raport - Zadanie numeryczne 9}

\begin{document}
	\maketitle

	\section{Wstęp teoretyczny}
	Funkcja rzeczywista ciągła na przedziale \([a;b]\) przyjmuje wszystkie wartości należące do tego przedziału.
	Korzystając z tego faktu, oraz wybierając \(a\) i \(b\) tak żeby znak funkcji \(f\) w tych punktach był różny, mamy pewność, że 
	\begin{align}
		\exists \,c \in [a;b] : f(c) = 0
	\end{align}
	Innymi słowy \(f\) ma co najmniej jeden pierwiastek w przedziale \([a;b]\).
	
	\subsection{}
	Na tym rozumowaniu opierają się metody falsi i bisekcji, służące znajdywaniu pierwiastków funkcji.
	Są to metody iteracyjne, sukcesywnie zawężające przedział \([a;b]\) zachowując przy tym różne znaki funkcji na krańcach tego przedziału.
	W obu metodach sprawdzany jest znak wybranego punktu \(c \in [a;b]\).
	Jeśli znak \(f(c)\) jest taki sam jak \(f(a)\) to \(c\) staje się nowym \(a\).
	W przeciwnym wypadku \(c\) staje się nowym \(b\). Metody rozróżnia sposób wyboru \(c\).
	Dla regula falsi jest to pierwiastek prostej przeprowadzonej przez f(a) i f(b):
	\begin{align}
		c = \frac{ f(a)b - f(b)a }{ f(a) - f(b) }
	\end{align}
	Natomiast dla metody bisekcji jest to średnia arytmetyczna a i b:
	\begin{align}
		c = f\left(\frac{a+b}{2}\right)
	\end{align}
	
	\subsection{}
	Inną strategię reprezentują metoda Newtona i metoda siecznych.
	Nie gwarantują one zbieżności, ale nie wymagają znajomosci przedziału na którym funkcja zmienia znak.
	Metoda Newtona uzyskuje następne przybliżenie znajdując pierwiastek prostej będącej styczną do funkcji \(f\) w poprzednim punkcie. Wymaga ona znajomości \(f'\):
	\begin{align}
		x_{n} = x_{n-1} - \frac{f(x_{n-1})}{f'(x_{n-1})}
	\end{align}
	Metoda siecznych działa analogicznie, jednak nie wymaga \(f'\), a prostą jest sieczna oparta na wartościach funkcji w dwóch poprzednich iteracjach:
	\begin{align}
		x_{n} = \frac{ f(x_{n-2})x_{n-1} - f(x_{n-1})x_{n-2} }{ f(x_{n-2}) - f(x_{n-1}) }
	\end{align}

	\subsection{}
	W zadaniu należy znaleźć pierwiastki następujących funkcji (6), (7). Z dokładnością do \(\epsilon=10^{-6}\), za pomocą opisanych wyżej metod.
	\begin{align}
		f(x) &= sin(x) - 0.37 \\
		g(x) &= f^2(x) = (sin(x) - 0.37)^2
	\end{align}
	Oraz zbadać zachowanie się błędu przybliżenia \(|x_i - x^\star|\) w zależności od iteracji.

	\section{Wyniki}

	\begin{figure}[H]
		\begin{center}
			\input{ch1.pgf}
		\end{center}
		\caption{Błąd przybliżenia pierwiastka od ilości iteracji dla funkcji \(f(x)\)}
		\label{ch1}
	\end{figure}
	\begin{figure}[H]
		\begin{center}
			\input{ch2.pgf}
		\end{center}
		\caption{Błąd przybliżenia pierwiastka od ilości iteracji dla funkcji \(g(x)\)}
		\label{ch2}
	\end{figure}
	\begin{figure}[H]
		\begin{center}
			\input{ch3.pgf}
		\end{center}
		\caption{Błąd przybliżenia pierwiastka od ilości iteracji dla funkcji \(u(x) = \frac{g(x)}{g'(x)}\)}
		\label{ch3}
	\end{figure}


	\section{Podsumowanie}
	Choć może być rozbieżna, dla pierwiastków jednokrotnych metoda Newtona uzyskuje zbieżność kwadratową, najlepszą pośród badanych metod.
	W przypadku pierwiastków wielokrotnych zbieżność metody staje się liniowa.
	W celu zachowania zbieżności kwadratowej, algorytm można usprawnić aby nie poszukiwał on \(x:g(x)=0\), tylko \(x:\frac{g(x)}{g'(x)}=0\).
	Zachowujemy w ten sposób pozycje miejsc zerowych, redukując jednocześnie ich krotność.
	To samo usprawnienie działa również w przypadku metody siecznych.

	Konieczność znania dwóch argumentów dla których funkcja przyjmuje różne znaki może uniemożliwiać poszukiwania pierwiastków za pomocą metod bisekcji i falsi.
	Nie jest więc możliwe ich zastosowanie w przypadku funkcji \(g\), ponieważ przyjmuje ona tylko wartości nieujemne.

	Pomimo tych samych ograniczeń metoda bisekcji ma dużo gorszą zbieżność od metody falsi, upraszcza jedynie wyznaczanie następnego punktu, nie wymagając przy tym dzielenia.

\end{document}
